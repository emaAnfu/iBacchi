% Chapter 1

\chapter{Templates} % Write in your own chapter title
\label{Chapter0} page header

Testo normale... \emph{corsivo} oppure \textbf{grassetto}.
Se vai a capo, non va a capo!!!

Per andare a capo devo lasciare una linea vuota, e inizia un nuovo paragrafo.

Se proprio vuoi andare a capo \\ fai così.

Elenco puntato:
\begin{itemize}
	\item primo elemento
	\item secondo elemento
\end{itemize}

Elenco numerato:
\begin{enumerate}
	\item primo elemento
	\item secondo elemento
\end{enumerate}

\section{Equazioni}
\label{sec:equazioni}

\subsection{Simboli matematici}

\subsubsection{Operatori}

Per scrivere simboli matematici nel testo usi i dollari. Per esempio $\alpha$, $y = x-3$ senza dollaro y = x-3. Altri esempi $x^2$, $x_1-x_2^3$, $y = e^{-\frac{x^2}{\sigma}}$.

\subsection{Equazioni grandi}

Se voglio scrivere un'equazione grande:

\begin{equation}
A = \begin{bmatrix}
		1 & 0 & 3 \\
		2 & 4 & 1 \\
		10 & 5 & 7 \\
		\end{bmatrix}
		\label{eq:matrice}
\end{equation}

\begin{equation}
y = \begin{cases}
		x^2-4, \quad \textrm{if } x \geq 0 \\
		-\frac{1}{x}, \quad \textrm{if } x < 0
		\end{cases}
		\label{eq:sistema}
\end{equation}

Come mostrato nella sezione \ref{sec:equazioni} e in eq. \eqref{eq:sistema}.

Qui sotto c'è la Fig. \ref{fig:logo}:
\begin{figure}[hbt!]
\centering
\includegraphics[width=0.5\textwidth]{figure/logounige.jpg}
\caption{Questa è la didascalia}
\label{fig:logo}
\end{figure}






